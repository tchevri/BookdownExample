\PassOptionsToPackage{unicode=true}{hyperref} % options for packages loaded elsewhere
\PassOptionsToPackage{hyphens}{url}
%
\documentclass[
  ignorenonframetext,
  noamsthm,11pt,a4paper,blue,xcolor=table]{beamer}
\usepackage{pgfpages}
\setbeamertemplate{caption}[numbered]
\setbeamertemplate{caption label separator}{: }
\setbeamercolor{caption name}{fg=normal text.fg}
\beamertemplatenavigationsymbolsempty
% Prevent slide breaks in the middle of a paragraph:
\widowpenalties 1 10000
\raggedbottom
\setbeamertemplate{part page}{
  \centering
  \begin{beamercolorbox}[sep=16pt,center]{part title}
    \usebeamerfont{part title}\insertpart\par
  \end{beamercolorbox}
}
\setbeamertemplate{section page}{
  \centering
  \begin{beamercolorbox}[sep=12pt,center]{part title}
    \usebeamerfont{section title}\insertsection\par
  \end{beamercolorbox}
}
\setbeamertemplate{subsection page}{
  \centering
  \begin{beamercolorbox}[sep=8pt,center]{part title}
    \usebeamerfont{subsection title}\insertsubsection\par
  \end{beamercolorbox}
}
\AtBeginPart{
  \frame{\partpage}
}
\AtBeginSection{
  \ifbibliography
  \else
    \frame{\sectionpage}
  \fi
}
\AtBeginSubsection{
  \frame{\subsectionpage}
}
\usepackage{lmodern}
\usepackage{amssymb,amsmath}
\usepackage{ifxetex,ifluatex}
\ifnum 0\ifxetex 1\fi\ifluatex 1\fi=0 % if pdftex
  \usepackage[T1]{fontenc}
  \usepackage[utf8]{inputenc}
  \usepackage{textcomp} % provides euro and other symbols
\else % if luatex or xelatex
  \usepackage{unicode-math}
  \defaultfontfeatures{Scale=MatchLowercase}
  \defaultfontfeatures[\rmfamily]{Ligatures=TeX,Scale=1}
\fi
% use upquote if available, for straight quotes in verbatim environments
\IfFileExists{upquote.sty}{\usepackage{upquote}}{}
\IfFileExists{microtype.sty}{% use microtype if available
  \usepackage[]{microtype}
  \UseMicrotypeSet[protrusion]{basicmath} % disable protrusion for tt fonts
}{}
\makeatletter
\@ifundefined{KOMAClassName}{% if non-KOMA class
  \IfFileExists{parskip.sty}{%
    \usepackage{parskip}
  }{% else
    \setlength{\parindent}{0pt}
    \setlength{\parskip}{6pt plus 2pt minus 1pt}}
}{% if KOMA class
  \KOMAoptions{parskip=half}}
\makeatother
\usepackage{xcolor}
\IfFileExists{xurl.sty}{\usepackage{xurl}}{} % add URL line breaks if available
\IfFileExists{bookmark.sty}{\usepackage{bookmark}}{\usepackage{hyperref}}
\hypersetup{
  pdftitle={A Minimal Book Example},
  pdfauthor={Yihui Xie},
  pdfborder={0 0 0},
  breaklinks=true}
\urlstyle{same}  % don't use monospace font for urls
\newif\ifbibliography
\usepackage{color}
\usepackage{fancyvrb}
\newcommand{\VerbBar}{|}
\newcommand{\VERB}{\Verb[commandchars=\\\{\}]}
\DefineVerbatimEnvironment{Highlighting}{Verbatim}{commandchars=\\\{\}}
% Add ',fontsize=\small' for more characters per line
\usepackage{framed}
\definecolor{shadecolor}{RGB}{248,248,248}
\newenvironment{Shaded}{\begin{snugshade}}{\end{snugshade}}
\newcommand{\AlertTok}[1]{\textcolor[rgb]{0.94,0.16,0.16}{#1}}
\newcommand{\AnnotationTok}[1]{\textcolor[rgb]{0.56,0.35,0.01}{\textbf{\textit{#1}}}}
\newcommand{\AttributeTok}[1]{\textcolor[rgb]{0.77,0.63,0.00}{#1}}
\newcommand{\BaseNTok}[1]{\textcolor[rgb]{0.00,0.00,0.81}{#1}}
\newcommand{\BuiltInTok}[1]{#1}
\newcommand{\CharTok}[1]{\textcolor[rgb]{0.31,0.60,0.02}{#1}}
\newcommand{\CommentTok}[1]{\textcolor[rgb]{0.56,0.35,0.01}{\textit{#1}}}
\newcommand{\CommentVarTok}[1]{\textcolor[rgb]{0.56,0.35,0.01}{\textbf{\textit{#1}}}}
\newcommand{\ConstantTok}[1]{\textcolor[rgb]{0.00,0.00,0.00}{#1}}
\newcommand{\ControlFlowTok}[1]{\textcolor[rgb]{0.13,0.29,0.53}{\textbf{#1}}}
\newcommand{\DataTypeTok}[1]{\textcolor[rgb]{0.13,0.29,0.53}{#1}}
\newcommand{\DecValTok}[1]{\textcolor[rgb]{0.00,0.00,0.81}{#1}}
\newcommand{\DocumentationTok}[1]{\textcolor[rgb]{0.56,0.35,0.01}{\textbf{\textit{#1}}}}
\newcommand{\ErrorTok}[1]{\textcolor[rgb]{0.64,0.00,0.00}{\textbf{#1}}}
\newcommand{\ExtensionTok}[1]{#1}
\newcommand{\FloatTok}[1]{\textcolor[rgb]{0.00,0.00,0.81}{#1}}
\newcommand{\FunctionTok}[1]{\textcolor[rgb]{0.00,0.00,0.00}{#1}}
\newcommand{\ImportTok}[1]{#1}
\newcommand{\InformationTok}[1]{\textcolor[rgb]{0.56,0.35,0.01}{\textbf{\textit{#1}}}}
\newcommand{\KeywordTok}[1]{\textcolor[rgb]{0.13,0.29,0.53}{\textbf{#1}}}
\newcommand{\NormalTok}[1]{#1}
\newcommand{\OperatorTok}[1]{\textcolor[rgb]{0.81,0.36,0.00}{\textbf{#1}}}
\newcommand{\OtherTok}[1]{\textcolor[rgb]{0.56,0.35,0.01}{#1}}
\newcommand{\PreprocessorTok}[1]{\textcolor[rgb]{0.56,0.35,0.01}{\textit{#1}}}
\newcommand{\RegionMarkerTok}[1]{#1}
\newcommand{\SpecialCharTok}[1]{\textcolor[rgb]{0.00,0.00,0.00}{#1}}
\newcommand{\SpecialStringTok}[1]{\textcolor[rgb]{0.31,0.60,0.02}{#1}}
\newcommand{\StringTok}[1]{\textcolor[rgb]{0.31,0.60,0.02}{#1}}
\newcommand{\VariableTok}[1]{\textcolor[rgb]{0.00,0.00,0.00}{#1}}
\newcommand{\VerbatimStringTok}[1]{\textcolor[rgb]{0.31,0.60,0.02}{#1}}
\newcommand{\WarningTok}[1]{\textcolor[rgb]{0.56,0.35,0.01}{\textbf{\textit{#1}}}}
\usepackage{longtable,booktabs}
\usepackage{caption}
% These lines are needed to make table captions work with longtable:
\makeatletter
\def\fnum@table{\tablename~\thetable}
\makeatother
\setlength{\emergencystretch}{3em}  % prevent overfull lines
\providecommand{\tightlist}{%
  \setlength{\itemsep}{0pt}\setlength{\parskip}{0pt}}
\setcounter{secnumdepth}{-2}

% set default figure placement to htbp
\makeatletter
\def\fps@figure{htbp}
\makeatother

%%%%%%%%%%%%%% Beamer-header.tex %%%%%%%%%%%%%%%%%%%%
\setlength{\heavyrulewidth}{1.5pt}
\setlength{\abovetopsep}{4pt}
\usepackage{array}
\usepackage{multirow}
\usepackage{multicol}  %% https://tex.stackexchange.com/questions/194426/split-itemize-into-multiple-columns
\setlength{\columnsep}{1cm}
\usepackage{makecell}
\usepackage{wrapfig}
\usepackage{float}
\usepackage{colortbl}
\usepackage{pdflscape}
\usepackage{tabu}
\usepackage{threeparttable}
%%% BELOW ALLOWS NICE COMPACT FORMAT BUT KILLS ALL HYPERLINKS!!
%\usepackage{pgfpages}
%\pgfpagesuselayout{2 on 1}[a4paper] %%% ALLOWS NICE COMPACT FORMAT BUT KILLS ALL HYPERLINKS!!
\mode<handout>{
\usepackage{pgfpages}
\pgfpagesuselayout{4 on 1}[a4paper,landscape] %%% ALLOWS NICE COMPACT FORMAT BUT KILLS ALL HYPERLINKS!!
}
% \usepackage{xcolor}  %% Commented out
% see https://tex.stackexchange.com/questions/83101/option-clash-for-package-xcolor
%\usepackage{etex}  % http://tex.stackexchange.com/questions/38607/no-room-for-a-new-dimen
\usepackage{adjustbox} % http://tex.stackexchange.com/questions/77998/fitting-tables-into-beamer
\usepackage{etoolbox}  %% Used for slide number section reset
\usepackage{graphicx, fancyvrb}
\usepackage{dsfont}
%\usepackage{tikz}
%\usetikzlibrary{arrows,shapes,backgrounds,fit,positioning,calc}
%\usepackage{tikzit}

%%%%%%%%%%%%%%%%%%%%%%%%%%%%%%%%%%%%%%%%%%%%%%%%%%%%%%%%%%
% Slide number section reset
% see https://tex.stackexchange.com/questions/278873/how-to-reset-frame-number-in-beamer-correctly-when-using-parts
%------------------------------------------------------------------------------
\makeatletter
\newcount\beamer@sectionstartframe
\beamer@sectionstartframe=1
\apptocmd{\beamer@section}{\addtocontents{nav}{\protect\headcommand{%
    \protect\beamer@sectionframes{\the\beamer@sectionstartframe}{\the\c@framenumber}}}}{}{}
\apptocmd{\beamer@section}{\beamer@sectionstartframe=\c@framenumber\advance\beamer@sectionstartframe by1\relax}{}{}
\AtEndDocument{\immediate\write\@auxout{\string\@writefile{nav}%
    {\noexpand\headcommand{\noexpand\beamer@sectionframes{\the\beamer@sectionstartframe}{\the\c@framenumber}}}}}{}{}
\def\beamer@startframeofsection{1}
\def\beamer@endframeofsection{1}
\def\beamer@sectionframes#1#2{%
    \ifnum\c@framenumber<#1%
    \else%
    \ifnum\c@framenumber>#2%
    \else%
    \gdef\beamer@startframeofsection{#1}%
    \gdef\beamer@endframeofsection{#2}%
    \fi%
    \fi%
}
\newcommand\insertsectionstartframe{\beamer@startframeofsection}
\newcommand\insertsectionendframe{\beamer@endframeofsection}
\makeatother

\def\inserttotalsectionframenumber{%
    \pgfmathparse{(\insertsectionendframe-\insertsectionstartframe+1)}%
    \pgfmathprintnumber[fixed,precision=2]{\pgfmathresult}%
}

\def\insertsectionframenumber{%
    \pgfmathparse{(\insertframenumber-\insertsectionstartframe+1)}%
    \pgfmathprintnumber[fixed,precision=2]{\pgfmathresult}%
}
%%%%%%%%%%%%%%%%%%%%%%%%%%%%%%%%%%%%%%%%%%%%%%%%%%%%%%%%%%
% Theme
%------
\usetheme{boxes} % https://tex.stackexchange.com/questions/234658/malmoe-section-in-footer
\usecolortheme{whale} % Used in Malmoe
\setbeamercolor*{titlelike}{parent=structure} % Used in Malmoe
% add page numbers for malmoe 
\addfootbox{title in head/foot}{\hfill\insertshortauthor\tiny\quad}
\addfootbox{author in head/foot}{%
    \tiny\quad\insertshorttitle%
    \hfill%
    \usebeamercolor[fg]{page number in head/foot}{%
        \insertsectionframenumber\,/\,\inserttotalsectionframenumber%
    }\tiny\quad}
\setbeamercolor{page number in head/foot}{fg=white}
% Get rid of headline / navigation symbols
%-----------------------------------------
\setbeamertemplate{headline}{}
%\setbeamercolor{frametitle}{bg=white}
%\setbeamertemplate{navigation symbols}{}  % This one removes navigation..
% Sections and subsections should not get their own damn slide
%-------------------------------------------------------------
\AtBeginSection{}
\AtBeginSubsection{}
%\AtBeginSubsubsection{}
% Reduce some of R Markdown’s odd vertical spacing
%-------------------------------------------------
\setlength{\emergencystretch}{0em}
\setlength{\parskip}{0pt}
\setlength{\heavyrulewidth}{1.5pt}
\setlength{\abovetopsep}{4pt}
%  Reset some default colors for itemize/enumerate/description environments
%--------------------------------------------------------------------------
\definecolor{darkred}{rgb}{0.7,0,0}
\definecolor{darkblue}{rgb}{0,0,0.8}
%  Used to reset basic black for itemize/enumerates within certain environments
%------------------------------------------------------------------------------
\setbeamercolor{description item}{fg=darkred!80!black}  %  Color of key word in desciption
\setbeamercolor{item}{fg=green}  %  The dot color
\setbeamercolor{itemize/enumerate body}{fg=black}    % Text Level 1
\setbeamercolor{itemize/enumerate subbody}{fg=darkblue}    % Text Level 2
\setbeamercolor{itemize/enumerate subsubbody}{fg=green!25!black}    % Text Level 3

\def\beghand{\mode<handout>}
\def\begpres{\mode<presentation>}
\def\begtab{\begin{tabular}}
\def\endtab{\end{tabular}}
\def\begenum{\begin{enumerate}}
\def\endenum{\end{enumerate}}
\def\begcols{\begin{columns}}
\def\begcol{\begin{column}}
\def\endcol{\end{column}}
\def\endcols{\end{columns}}
\def\begonlyhandout{\begin{onlyenv}<handout:0>}
\def\endonlyhandout{\end{onlyenv}}

\institute{My institution}
\newcommand{\CourseName}{My course}
\newcommand{\Yr}{, 2020}
%%%%%%%%%%%%%%%%%%%%%%%%%%%%%%%%%%%%%%%%%%%%%%%%%%%%%%%%%%

\title{A Minimal Book Example}
\author{Yihui Xie}
\date{2020-08-26}

\begin{document}
\frame{\titlepage}

\mode<all>
\mode<presentation>{
        \title[\CourseName{} Syllabus]{\CourseName{} \newline \, Syllabus}
        \date{1 August\Yr{}}
        }

\hypertarget{prerequisites}{%
\section{Prerequisites}\label{prerequisites}}

\begin{frame}[fragile]{Bookdown Markdown}
\protect\hypertarget{bookdown-markdown}{}

\begin{itemize}
\item
  This is a \emph{sample} book written in \textbf{Markdown}. You can use anything that Pandoc's Markdown supports, e.g., a math equation \(a^2 + b^2 = c^2\).
\item
  The \textbf{bookdown} package can be installed from CRAN or Github:

\begin{Shaded}
\begin{Highlighting}[]
\KeywordTok{install.packages}\NormalTok{(}\StringTok{"bookdown"}\NormalTok{)}
\CommentTok{# or the development version}
\CommentTok{# devtools::install_github("rstudio/bookdown")}
\end{Highlighting}
\end{Shaded}
\item
  Remember each Rmd file contains one and only one chapter, and a chapter is defined by the first-level heading \texttt{\#}.
\item
  To compile this example to PDF, you need XeLaTeX. You are recommended to install TinyTeX (which includes XeLaTeX): \url{https://yihui.org/tinytex/}.
\end{itemize}

\end{frame}

\hypertarget{intro}{%
\section{Introduction}\label{intro}}

\mode<presentation> 
{ 
\title[Lecture 1]{\CourseName{} \newline \, Lecture 1}
\date{1 September\Yr{}}
}

\begin{frame}{}
\protect\hypertarget{section}{}

\titlepage

\end{frame}

\begin{frame}[fragile]{Figure}
\protect\hypertarget{figure}{}

\begin{Shaded}
\begin{Highlighting}[]
\KeywordTok{par}\NormalTok{(}\DataTypeTok{mar =} \KeywordTok{c}\NormalTok{(}\DecValTok{4}\NormalTok{, }\DecValTok{4}\NormalTok{, }\FloatTok{.1}\NormalTok{, }\FloatTok{.1}\NormalTok{))}
\KeywordTok{plot}\NormalTok{(pressure, }\DataTypeTok{type =} \StringTok{'b'}\NormalTok{, }\DataTypeTok{pch =} \DecValTok{19}\NormalTok{)}
\end{Highlighting}
\end{Shaded}

\begin{figure}

{\centering \includegraphics[width=0.8\linewidth]{BookdownExample_files/figure-beamer/nice-fig-1} 

}

\caption{Here is a nice figure!}\label{fig:nice-fig}
\end{figure}

\end{frame}

\begin{frame}[fragile]{Table}
\protect\hypertarget{table}{}

\begin{Shaded}
\begin{Highlighting}[]
\NormalTok{knitr}\OperatorTok{::}\KeywordTok{kable}\NormalTok{(}
  \KeywordTok{head}\NormalTok{(iris, }\DecValTok{20}\NormalTok{), }\DataTypeTok{caption =} \StringTok{'Here is a nice table!'}\NormalTok{,}
  \DataTypeTok{booktabs =} \OtherTok{TRUE}
\NormalTok{)}
\end{Highlighting}
\end{Shaded}

\begin{table}

\caption{\label{tab:nice-tab}Here is a nice table!}
\centering
\begin{tabular}[t]{rrrrl}
\toprule
Sepal.Length & Sepal.Width & Petal.Length & Petal.Width & Species\\
\midrule
5.1 & 3.5 & 1.4 & 0.2 & setosa\\
4.9 & 3.0 & 1.4 & 0.2 & setosa\\
4.7 & 3.2 & 1.3 & 0.2 & setosa\\
4.6 & 3.1 & 1.5 & 0.2 & setosa\\
5.0 & 3.6 & 1.4 & 0.2 & setosa\\
\addlinespace
5.4 & 3.9 & 1.7 & 0.4 & setosa\\
4.6 & 3.4 & 1.4 & 0.3 & setosa\\
5.0 & 3.4 & 1.5 & 0.2 & setosa\\
4.4 & 2.9 & 1.4 & 0.2 & setosa\\
4.9 & 3.1 & 1.5 & 0.1 & setosa\\
\addlinespace
5.4 & 3.7 & 1.5 & 0.2 & setosa\\
4.8 & 3.4 & 1.6 & 0.2 & setosa\\
4.8 & 3.0 & 1.4 & 0.1 & setosa\\
4.3 & 3.0 & 1.1 & 0.1 & setosa\\
5.8 & 4.0 & 1.2 & 0.2 & setosa\\
\addlinespace
5.7 & 4.4 & 1.5 & 0.4 & setosa\\
5.4 & 3.9 & 1.3 & 0.4 & setosa\\
5.1 & 3.5 & 1.4 & 0.3 & setosa\\
5.7 & 3.8 & 1.7 & 0.3 & setosa\\
5.1 & 3.8 & 1.5 & 0.3 & setosa\\
\bottomrule
\end{tabular}
\end{table}

\begin{itemize}
\tightlist
\item
  You can write citations, too. For example, we are using the \textbf{bookdown} package (Xie \protect\hyperlink{ref-R-bookdown}{2020}) in this sample book, which was built on top of R Markdown and \textbf{knitr} (Xie \protect\hyperlink{ref-xie2015}{2015}).
\end{itemize}

\end{frame}

\hypertarget{literature}{%
\section{Literature}\label{literature}}

\begin{frame}{Lit Review}
\protect\hypertarget{lit-review}{}

Here is a review of existing methods.

\end{frame}

\hypertarget{methods}{%
\section{Methods}\label{methods}}

We describe our methods in this chapter.

\hypertarget{applications}{%
\section{Applications}\label{applications}}

Some \emph{significant} applications are demonstrated in this chapter.

\begin{frame}{Example one}
\protect\hypertarget{example-one}{}

\end{frame}

\begin{frame}{Example two}
\protect\hypertarget{example-two}{}

\end{frame}

\hypertarget{final-words}{%
\section{Final Words}\label{final-words}}

We have finished a nice book.

\hypertarget{refs}{}
\leavevmode\hypertarget{ref-xie2015}{}%
Xie, Yihui. 2015. \emph{Dynamic Documents with R and Knitr}. 2nd ed. Boca Raton, Florida: Chapman; Hall/CRC. \url{http://yihui.org/knitr/}.

\leavevmode\hypertarget{ref-R-bookdown}{}%
---------. 2020. \emph{Bookdown: Authoring Books and Technical Documents with R Markdown}. \url{https://CRAN.R-project.org/package=bookdown}.

\mode*

\end{document}
